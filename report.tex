\documentclass[a4paper, 12pt, twoside, openright]{book}
\usepackage[italian]{babel}
\usepackage[T1]{fontenc}
\usepackage[utf8]{inputenc}
\usepackage{fancyhdr}
\usepackage{float}
\usepackage{graphicx}
\usepackage{wrapfig}
\usepackage{setspace}

%------------------------------ colors
\usepackage[usenames,dvipsnames,table]{xcolor} % use colors on table and more
\definecolor{333}{RGB}{51, 51, 51} % define custom color
%------------------------------ source code
\usepackage{listings}
\lstset{
  basicstyle=\footnotesize\sffamily,
  commentstyle=\itshape\color{gray},
  captionpos=b,
  frame=shadowbox,
  language=HTML,
  rulesepcolor=\color{333},
  tabsize=2
}
%------------------------------ define Abstract environment, missing in the 'book' class
\newenvironment{abstract}{\cleardoublepage \null \vfill \begin{center}\bfseries\abstractname \end{center}}{\vfill\null}
\addto\captionsenglish{\renewcommand*\abstractname{Sommario}} % change Abstract title
%------------------------------ active url
\usepackage{url}
\renewcommand{\UrlFont}{\color{black}\small\ttfamily}
\usepackage[colorlinks=true, linkcolor=black, citecolor=black, urlcolor=black]{hyperref} % active ref
%------------------------------ macros
\newcommand{\sectionname}{Section} % define Section ref
\newcommand{\subsectionname}{Sub-section} % define Sub-section ref
\renewcommand*\arraystretch{1.4} % tables padding


\begin{document}
\frontmatter
\begin{titlepage}
	%immagini di intestazione
	\begin{flushleft}
		
		\begin{minipage}[b][1,8 cm][c]{0.3\columnwidth}
			\textsf{{\color{Sepia}{DIPARTIMENTO\\DI INGEGNERIA\\DELL'INFORMAZIONE}}}
		\end{minipage}
	\end{flushleft}
	
	%titoli vari
	\vfill
	\begin{center}
		\begin{large}
			DIPARTIMENTO DI INGEGNERIA DELL'INFORMAZIONE
			\\~\\
			CORSO DI LAUREA IN INGEGNERIA INFORMATICA
			\\~\\~\\
			TITOLO
		\end{large}
	\end{center}
\end{titlepage}
\cleardoublepage % make left page blank

\begingroup %------------------------------ CONTENTS
  \makeatletter
  \let\ps@plain\ps@empty
  \makeatother
  \tableofcontents
  \clearpage
\endgroup


\begin{abstract} %------------------------------ ABSTRACT
\markboth{}{} % remove header
\thispagestyle{empty}
\end{abstract}


\mainmatter\doublespace 

\chapter*{Introduzione} %------------------------------ INTRODUCTION
\thispagestyle{empty}

\chapter{Strumenti Web utilizzati nelle PWA}
Questa sezione copre le tecnologie web usate nelle Web App che non sono vincolate a uno specifico framework di sviluppo: di esse si fornirà una descrizione delle relative funzionalità e di una loro possibile implementazione in JavaScript.\\
Alcuni screenshot di questa sezione sono stati catturati dai Developer Tools (DevTools) del browser: per visualizzarli su PC è sufficiente premere F12 (valido per Firefox e per qualunque browser Chromium-based come Google Chrome o Microsoft Edge). Le immagini fanno in particolare riferimento ai DevTools di Google Chrome. Per quanto riguarda DevTools per Chromium-based browsers, gli screenshot sono stati tutti tratti dalla sezione \texttt{Application}, accessibile dalla riga di menù collocata in alto (se non è immediatamente visibile, allora potrebbe essere stata compressa all'interno del pulsante "\texttt{>>}"). 

\section{Cookie} %<-----------MANCA INTRODUZIONE, CAPIENZA MASSIMA DEI COOKIE E IMMAGINI
Un \textit{Cookie} è una stringa di testo che viene inviata dal server web al client, il quale dovrà poi memorizzarla e reinviarla al server senza modifiche ogni volta che accede alla stessa porzione di uno stesso dominio web\cite{Wiki:cookies}. I Cookie possono avere diversi utilizzi, in quanto permettono di identificare un utente durante la sua navigazione nel sito internet e perché permettono di definire uno stato per le pagine web: essi sono ad esempio usati in buona parte delle operazioni di login, infatti senza di essi non vi sarebbe alcuna differenza fra una pagina caricata prima di effettuare l'accesso da una pagina caricata dopo. L'applicazione più nota dei Cookie è quella di tracciare la navigazione degli utenti per costruire un profilo personalizzato utile a fornire annunci personalizzati: i domini inserzionistici infatti caricano le loro pubblicità su diversi siti e ogni volta che l'utente visita un sito e carica una loro pubblicità viene installato un Cookie. Se poi l'utente visita un'altra pagina contenente annunci da quel dominio essop sarà in grado di capirlo grazie al nuovo cookie che invierà.\\
I Cookie hanno una dimensione ridotta, infatti, dato che il client può dover inoltrare anche centinaia di Cookie durante la navigazione, delle dimensioni eccessive provocherebbero a danni alle prestazioni. Essi sono inviati attraverso specifici Header del protocollo HTTP: nel caso di textit{HTTP resonse} viene usato l'header \texttt{Set-Cookie} mentre per la \textit{HTTP request} si usa \texttt{Cookie}. A un Cookie viene associata inoltre una data di scadenza oltre la quale non viene considerato più valido.\\

%COOKIE TECNICO O DI CONSENSO
\subsection{implementazione}
I Cookie hanno un'interfaccia molto primitiva: non sono definite, infatti, delle funzioni per l'aggiunta, rimozione o la modifica di Cookie. L'unico modo per inserire, modificare, leggere ed eliminare Cookie è mediante l'attributo \texttt{document.cookie}\cite{MDN_Web_docs:cookies}.\\
Per inserire un nuovo Cookie basta assegnare una nuova stringa a \linebreak\texttt{document.cookie}\cite{MDN_Web_docs:cookies}: tale stringa dovrà presentarsi nel formato \linebreak\texttt{"name=value; optionalField1=optionalValue1; optionalField2=\linebreak optionalValue2;..."}\cite{MDN_Web_docs:cookies}. La coppia \texttt{"name=value"} deve essere specificata, altrimenti l'inserimento fallirà silenziosamente, tutte le coppie successive sono invece opzionali. I campi opzionali sono i seguenti:
\begin{itemize}
\item\texttt{"expires="}: definisce la data di scadenza del Cookie come stringa UTC; per esprimere una data in tale formato è possibile usare il metodo \texttt{toUTCString()} della classe \texttt{Date} definita in JavaScript\cite{MDN_Web_docs:cookies}. Se non sono specificati né \texttt{expires} né \texttt{max-age} allora il Cookie scadrà al termine della sessione\cite{MDN_Web_docs:cookies}.
\item\texttt{"max-age="}: specifica la durata del Cookie in secondi\cite{MDN_Web_docs:cookies}.
\item\texttt{"secure"}: indica che il Cookie deve essere trasmesso solo attraverso un protocollo sicuro\cite{MDN_Web_docs:cookies}.
\item\texttt{"partitioned"}: indica che il Cookie deve essere memorizzato in memoria partizionata\cite{MDN_Web_docs:cookies}. Si Supponga di accedere a un sito \texttt{A} che carica contenuti da un sito di terze parti. Al momento del caricamento quest'ultimo imposta un Cookie sul dispositivo dell'utente. Si supponga ora di spostarsi a un sito \texttt{B}, che carica anch'esso contenuti dallo stesso sito di prima. Se il Cookie non è partizionato, allora il sito di terze parti sarà in grado di accedere al Cookie definito precedentemente, difatti, in questo caso, la chiave del Cookie è definita solo dal suo host. Se, invece, il Cookie è partizionato, allora il sito di terze parti non sarà in grado di accedere al Cookie definito durante la navigazione in \texttt{A} in quanto, in questo caso, la sua chiave è definita dalla coppia host + sito in cui è caricato il contenuto\cite{MDN_Web_docs:CHIPS}. Un partitioned Cookie permette di garantire maggiore sicurezza, dato che impedisce il tracciamento \textit{cross-site} dell'utente\cite{MDN_Web_docs:CHIPS}.
\item\texttt{"domain="}: specifica il dominio a cui il Cookie potrà essere inviato; se non inserito allora assume un valore di default che coincide con l'host del documento. Si ha inoltre che il Cookie è visibile ai sottodomini solo quando questo parametro è esplicitato\cite{MDN_Web_docs:cookies}.
\item\texttt{"path="}: specifica il percorso del dominio in cui il Cookie è visibile; il Cookie potrà essere inviato solo alla porzione indicata da path all'iinterno del dominio specificato; sono incluse anche eventuali subdirectory\cite{MDN_Web_docs:HTTP_cookies}. Se non inserito allora assume il valore di default \texttt{"/"} (la root directory). Il path e il domain definiscono assieme l'ambito di visibilità dei Coookie\cite{Wiki:cookies}.
\item\texttt{"samesite="}: definisce quando inviare il Cookie al server\cite{MDN_Web_docs:cookies}. Esso può assumere valore \texttt{lax} se il Cookie può essere inviato solo in occasione di \textit{same-site requests} e \textit{top-level navigation requests}\footnote{cioè una navigazione a un altro sito che porta alla modifica del contenuto della barra degli indirizzi\cite{Stack_overflow:samesite}}\cite{MDN_Web_docs:cookies} (in questo secondo caso, però, il Cookie può essere inviato solo attraverso \textit{safe requests}, come \texttt{GET} o \texttt{HEAD} ma non \texttt{POST}\cite{Stack_overflow:samesite}), \texttt{strict} se si vuole impedire l'invio del Cookie attraverso cross-site requests \cite{MDN_Web_docs:cookies}, \texttt{none} se non si applica alcun vincolo\cite{MDN_Web_docs:cookies} (in quest'ultimo caso è però richiesto che sia esplicitato il parametro \texttt{secure}\cite{MDN_Web_docs:HTTP_cookies}).
\end{itemize}
In DevTools è possibile mostrare una visuale dettagliata di tutti i Cookie installati, con anche la possibilità di eliminare quelli indesiderati: per browser Chromium-based basta cliccare su "Application" e poi "Cookies" nel menù a sinistra, per Firefox invece "Archiviazione" e poi "Cookie".\\
Una volta inserito un Cookie esso non può essere modificato direttamente; è possibile solo sostituire questo con un altro: per farlo basta assegnare a \texttt{document.cookie} un nuovo Cookie con lo stesso nome di quello da sovrascrivere\cite{W3Schools:cookies}.\\
Per quanto riguarda l'eliminazione dei Cookie, l'unica strategia disponibile è quella di sovrascrivere il Cookie con un altro avente scadenza già passata\cite{W3Schools:cookies}.\\
L'attributo \texttt{document.cookie} può anche essere acceduto in lettura: in tal caso si ottiene una lista delle sole coppie nome-valore di tutti i Cookie salvati in quel momento\cite{MDN_Web_docs:cookies}. Per visualizzare tutti gli altri parametri è necessario farlo da DevTools.
\backmatter

\begingroup %------------------------------ BIBLIOGRAPHY
  \makeatletter
  \let\ps@plain\ps@empty
  \makeatother
  \bibliography{report}
  \addcontentsline{toc}{chapter}{Bibliografia}
  \bibliographystyle{ieeetr} % sort in order of appearance
\endgroup
\end{document} 