\documentclass[a4paper, 12pt, twoside, openright]{book}
\usepackage[italian]{babel}
\usepackage[T1]{fontenc}
\usepackage[utf8]{inputenc}
\usepackage{fancyhdr}
\usepackage{float}
\usepackage{graphicx}
\usepackage{wrapfig}
\usepackage{setspace}

%------------------------------ colors
\usepackage[usenames,dvipsnames,table]{xcolor} % use colors on table and more
\definecolor{333}{RGB}{51, 51, 51} % define custom color
%------------------------------ source code
\usepackage{listings}
\lstset{
  basicstyle=\footnotesize\sffamily,
  commentstyle=\itshape\color{gray},
  captionpos=b,
  frame=shadowbox,
  language=HTML,
  rulesepcolor=\color{333},
  tabsize=2
}
%------------------------------ define Abstract environment, missing in the 'book' class
\newenvironment{abstract}{\cleardoublepage \null \vfill \begin{center}\bfseries\abstractname \end{center}}{\vfill\null}
\addto\captionsenglish{\renewcommand*\abstractname{Sommario}} % change Abstract title
%------------------------------ active url
\usepackage{url}
\renewcommand{\UrlFont}{\color{black}\small\ttfamily}
\usepackage[colorlinks=true, linkcolor=black, citecolor=black, urlcolor=black]{hyperref} % active ref
%------------------------------ macros
\newcommand{\sectionname}{Section} % define Section ref
\newcommand{\subsectionname}{Sub-section} % define Sub-section ref
\renewcommand*\arraystretch{1.4} % tables padding


\begin{document}
\frontmatter
\begin{titlepage}
	%immagini di intestazione
	\begin{flushleft}
		
		\begin{minipage}[b][1,8 cm][c]{0.3\columnwidth}
			\textsf{{\color{Sepia}{DIPARTIMENTO\\DI INGEGNERIA\\DELL'INFORMAZIONE}}}
		\end{minipage}
	\end{flushleft}
	
	%titoli vari
	\vfill
	\begin{center}
		\begin{large}
			DIPARTIMENTO DI INGEGNERIA DELL'INFORMAZIONE
			\\~\\
			CORSO DI LAUREA IN INGEGNERIA INFORMATICA
			\\~\\~\\
			TITOLO
		\end{large}
	\end{center}
\end{titlepage}
\cleardoublepage % make left page blank

\begingroup %------------------------------ CONTENTS
  \makeatletter
  \let\ps@plain\ps@empty
  \makeatother
  \tableofcontents
  \clearpage
\endgroup


\begin{abstract} %------------------------------ ABSTRACT
\markboth{}{} % remove header
\thispagestyle{empty}
\end{abstract}


\mainmatter\doublespace 

\chapter*{Introduzione} %------------------------------ INTRODUCTION
\thispagestyle{empty}

\chapter{Strumenti Web utilizzati nelle PWA}
%SERVONO ESEMPI?
Questa sezione copre le tecnologie web usate nelle Web App che non sono vincolate a uno specifico framework di sviluppo: di esse si fornirà una descrizione delle relative funzionalità e di una loro possibile implementazione in JavaScript.\\
Alcuni screenshot di questa sezione sono stati catturati dai Developer Tools (DevTools) del browser: per visualizzarli su PC è sufficiente premere F12 (valido per Firefox e per qualunque browser Chromium-based come Google Chrome o Microsoft Edge). Le immagini fanno in particolare riferimento ai DevTools di Google Chrome.

\section{Cookie}
Un \textit{Cookie} è una stringa di testo che viene inviata dal server web al client, il quale avrà poi il compito di memorizzarla e reinviarla al server, senza modifiche, ogni volta che accede alla stessa porzione di uno stesso dominio web \cite{Wiki:cookies}.\\
I Cookie hanno una dimensione ridotta: infatti, dato che il browser può dover inoltrare anche centinaia di Cookie durante la navigazione, delle dimensioni eccessive provocherebbero danni alle prestazioni. Essi sono inviati attraverso specifici header del protocollo \textit{HTTP}: nel caso di \textit{HTTP response} viene usato l'header \texttt{Set-Cookie} mentre per la \textit{HTTP request} si usa \texttt{Cookie} \cite{Wiki:cookies}. A un Cookie viene associata inoltre una data di scadenza, oltre la quale non viene considerato più valido \cite{Wiki:cookies}.\\
I Cookie possono avere diversi scopi: possono coprire funzionalità necessarie al corretto funzionamento del sito (in tal caso si parla di \textit{cookie tecnici}), possono raccogliere dati in forma anonima a fini statistici (\textit{cookie statistici}) oppure possono tracciare la navigazione dell'utente, con lo scopo di costruire un profilo personalizzato del cliente utile a fornire annunci mirati (\textit{cookie pubblicitari}) \cite{Wiki:cookies}. Questi ultimi sono stati oggetto di diverse discussioni, in quanto un loro abuso potrebbe costituire una minaccia alla privacy degli utenti; per questo motivo il loro utilizzo è normato da svariate leggi in diversi stati: in Europa il \textit{GDPR} (\textit{General Data Protectioni Regulation}) ha stabilito che tutti i siti internet che fanno uso di Cookie pubblicitari di terze parti sono obbligati a comunicarlo all'utente all'inizio della sua navigazione nella pagina web \cite{Cookiebot}. Tale comunicazione deve essere effettuata attraverso un banner che ostacoli la visualizzazione della pagina: in esso devono essere elencati tutti i Cookie pubblicitari presenti nel sito, specificando anche chi elaborerà i dati raccolti e con quali finalità \cite{Cookiebot}; è inoltre obbligatorio fornire la possibilità di negare il consenso dei singoli Cookie, non è sufficiente permettere all'utente di negare/consentire tutti i Cookie in blocco \cite{Cookiebot}. Un Cookie non potrà essere attivato se non dopo il consenso esplicito da parte dell'utente, il quale sarà registrato in opportuna documantazione a testimoniare alle autorità che l'autorizzazione è stata concessa \cite{Cookiebot}. Il consenso a un Cookie deve essere fornito mediante un'azione non equivoca (come ad esempio cliccare sul pulsante "Acconsento"); azioni come continuare la navigazione sul sito non devono essere considerate come permesso per installare i Cookie \cite{Cookiebot}. Si deve inoltre fornire la possibilità di modificare le proprie scelte in un secondo momento. All'inizio della navigazione tutti i Cookie di terze parti e non strettamente necessari al corretto funzionamento del servizio devono essere preventivamente disattivati in attesa della scelta dell'utente. Una volta confermata la scelta il \textit{Cookie di consenso} si occuperà di attivare tutti i soli Cookie inserzionistici per il quale l'utente ha dato esplicito consenso \cite{Cookiebot}.\\
Diversi siti internet fanno uso dei cookie come identificatore di sessione univoco dell'utente (utile ad esempio per permettere al cliente di mantenere l'accesso senza dover fare continuamente login): questo può costituire un rischio per la sicurezza, in quanto un utente malintenzionato può rubare il cookie di qualcun altro e sfruttarlo per impersonare la vittima. Questo problema può essere risolto sfruttando soli Cookie con la flag "\texttt{Secure}" (che consente l'invio del Cookie solo attraverso protocollo criptato HTTPS) \cite{Wiki:cookies}.
\subsection{implementazione}
I Cookie hanno un'interfaccia molto primitiva: non sono definite, infatti, delle funzioni per l'aggiunta, rimozione o la modifica di essi. L'unico modo per inserire, modificare, leggere ed eliminare Cookie è mediante l'attributo \texttt{document.cookie} \cite{MDN_Web_docs:cookies}.\\
Per inserire un nuovo Cookie basta assegnare una nuova stringa a \linebreak\texttt{document.cookie} \cite{MDN_Web_docs:cookies}: tale stringa dovrà presentarsi nel formato \linebreak\texttt{"name=value; optionalField1=optionalValue1; optionalField2=\linebreak optionalValue2;..."} \cite{MDN_Web_docs:cookies}. La coppia \texttt{"name=value"} deve essere specificata, altrimenti l'inserimento fallirà silenziosamente, tutte le coppie successive sono invece opzionali. I campi opzionali sono i seguenti:
\begin{itemize}
\item\texttt{"expires="}: definisce la data di scadenza del Cookie come stringa UTC; per esprimere una data in tale formato è possibile usare il metodo \texttt{toUTCString()} della classe \texttt{Date} definita in JavaScript \cite{MDN_Web_docs:cookies}. Se non sono specificati né \texttt{expires} né \texttt{max-age} allora il Cookie scadrà al termine della sessione \cite{MDN_Web_docs:cookies}.
\item\texttt{"max-age="}: specifica la durata del Cookie in secondi \cite{MDN_Web_docs:cookies}.
\item\texttt{"secure"}: indica che il Cookie può essere trasmesso solo attraverso il protocollo criptato HTTPS \cite{MDN_Web_docs:cookies}.
\item\texttt{"partitioned"}: indica che il Cookie deve essere memorizzato in memoria partizionata \cite{MDN_Web_docs:cookies}. Un Cookie partizionato impedisce il tracciamento cross-site dell'utente, meccanismo che viene ad esempio usato dagli inserzionisti per costruire un profilo personalizzato dell'utente utile a fornire pubblicità mirata \cite{MDN_Web_docs:CHIPS}. Si Supponga di accedere a un sito \texttt{A} che carica un annuncio da un sito di terze parti. Al momento del caricamento, quest'ultimo imposta un Cookie sul dispositivo dell'utente\footnote{supponendo che l'utente abbia acconsentito a ciò}. Si supponga ora di spostarsi a un sito \texttt{B}, che carica lo stesso annuncio di prima. Se il Cookie non è partizionato, allora il sito di terze parti sarà in grado di accedere al Cookie definito precedentemente, difatti, in questo caso, la chiave del Cookie è definita solo dall'host che lo ha impostato; il dominio da cui proviene l'inserzione è pertanto in grado di capire che l'utente ha visitato sia \texttt{A} che \texttt{B}. Se, invece, il Cookie è partizionato, allora il sito di terze parti non sarà in grado di accedere al Cookie definito durante la navigazione in \texttt{A}, in quanto, in questo caso, la sua chiave è definita dalla coppia host + sito in cui è caricato il contenuto \cite{MDN_Web_docs:CHIPS}.
\item\texttt{"domain="}: specifica il dominio a cui il Cookie potrà essere inviato; se non inserito allora assume un valore di default che coincide con l'host della pagina. Si ha inoltre che il Cookie è visibile ai sottodomini solo quando questo parametro è esplicitato \cite{MDN_Web_docs:cookies}.
\item\texttt{"path="}: specifica il percorso del dominio in cui il Cookie è visibile; il Cookie potrà essere inviato solo alla porzione indicata da \texttt{path} all'interno del dominio specificato; sono incluse anche eventuali sottodirectory \cite{MDN_Web_docs:HTTP_cookies}. Se non inserito allora assume il valore di default \texttt{"/"} (la \textit{root directory}). \texttt{path} e \texttt{domain} definiscono assieme l'ambito di visibilità del Coookie \cite{Wiki:cookies}.
\item\texttt{"samesite="}: definisce quando inviare il Cookie al server \cite{MDN_Web_docs:cookies}. Esso può assumere valore \texttt{lax} se il Cookie può essere inviato solo in occasione di \textit{same-site requests} e \textit{top-level navigation requests}\footnote{cioè una navigazione a un altro sito che porta alla modifica del contenuto della barra degli indirizzi \cite{Stack_overflow:samesite}} \cite{MDN_Web_docs:cookies} (in questo secondo caso, però, il Cookie può essere inviato solo attraverso \textit{safe requests}, come \texttt{GET} o \texttt{HEAD} ma non \texttt{POST} \cite{Stack_overflow:samesite}), \texttt{strict} se si vuole impedire l'invio del Cookie attraverso cross-site requests \cite{MDN_Web_docs:cookies}, \texttt{none} se non si applica alcun vincolo \cite{MDN_Web_docs:cookies} (in quest'ultimo caso è però richiesto che sia esplicitato il parametro \texttt{secure} \cite{MDN_Web_docs:HTTP_cookies}).
\end{itemize}
In DevTools è possibile mostrare una visuale dettagliata di tutti i Cookie installati, con anche la possibilità di eliminare quelli indesiderati: per browser Chromium-based basta cliccare su "\texttt{Application}" e poi "\texttt{Cookies}" nel menù a sinistra, per Firefox invece "\texttt{Storage}" e poi "\texttt{Cookies}".\\
Una volta inserito un Cookie esso non può essere modificato direttamente; è possibile solo sostituire questo con un altro: per farlo basta assegnare a \texttt{document.cookie} un nuovo Cookie con lo stesso nome di quello da sovrascrivere \cite{W3Schools:cookies}.\\
L'esempio in \figurename~\ref{example_cookie:code_snippet} costruisce un Cookie con nome "matricola", valore "123456", scadenza l'1 gennaio 2026 e che può essere trasmesso solo attraverso same-site request con protocollo HTTPS, come mostrato in \figurename~\ref{example_cookie}.\\
\begin{figure}[ht]
  \centering
  \includegraphics[width=15cm]{images/cookies/code_snippet.png}
  \caption{Esempio di creazione di un Cookie}
  \label{example_cookie:code_snippet}
\end{figure}
\begin{figure}[ht]
  \centering
  \includegraphics[height=7cm]{images/cookies/demonstration.png}
  \caption{Il Cookie creato prima}
  \label{example_cookie}
\end{figure}
Per quanto riguarda l'eliminazione dei Cookie, l'unica strategia disponibile è quella di sovrascrivere il Cookie con un altro avente scadenza già passata \cite{W3Schools:cookies}.\\
L'attributo \texttt{document.cookie} può anche essere acceduto in lettura: in tal caso si ottiene una lista delle sole coppie nome-valore di tutti i Cookie salvati in quel momento \cite{MDN_Web_docs:cookies}. Tutti gli altri parametri possono essere visualizzati solo da DevTools.

\section{Local Storage e Session Storage} %<-----MANCANO ESEMPI
\texttt{\textit{localStorage}} e \texttt{\textit{sessionStorage}} sono entrambi oggetti di tipo \texttt{Storage}: in quanto tali, essi permettono di salvare dati in memoria locale sottoforma di coppie chiave-valore (entrambe stringhe) \cite{MDN_Web_docs:storage_API}.\\
I due oggetti differscono per visibilità e durata: \texttt{localStorage} serve a contenere dati permanenti (cioè che rimangono in memoria indefinitamente, salvo esplicita rimozione da parte dell'utente o del programmatore) e condivisi fra le varie schede, per cui schede distinte del medesimo browser connesse allo stesso sito condividono lo stesso \texttt{localStorage} e, pertanto, modifiche apportate da una pagina sono visibili anche nell'altra pagina \cite{MDN_Web_docs:localStorage}. D'altro canto \texttt{sessionStorage} è progettato per contenere dati relativi alla singola sessione di navigazione: esso, quindi, non è condiviso fra schede e viene rimosso automaticamente alla chiusura della pagina (viene però conservato quando la pagina è ricaricata) \cite{MDN_Web_docs:sessionStorage}.\\
DevTools fornisce la possibilità di visualizzare \texttt{localStorage} e \texttt{sessionStorage} di una pagina: per browser Chromium-based è sufficiente cliccare su "\texttt{Application}" e poi "\texttt{Local Storage}" o "\texttt{Session Storage}" mentre per Firefox si deve cliccare su "\texttt{Storage}", quindi "\texttt{Local Storage}" o "\texttt{Session Storage}".
\subsection{implementazione}
Dato che \texttt{localStorage} e \texttt{sessionStorage} sono entrambi istanze di \texttt{Storage}, essi condividono la stessa API, composta dai metodi
\begin{itemize}
\item\texttt{setItem(key, value)}, che inserisce una nuova coppia chiave-valore in \texttt{Storage} se la chiave non è presente, altrimenti sostituisce il precedente valore a essa associato con il nuovo \cite{MDN_Web_docs:localStorage}.
\item\texttt{getItem(key)}, che restituisce il valore associato a \texttt{key} \cite{MDN_Web_docs:localStorage}.
\item\texttt{removeItem(key)}, che rimuove la coppia con chiave \texttt{key} \cite{MDN_Web_docs:localStorage}.
\item\texttt{clear()}, che svuota l'intero \texttt{Storage} \cite{MDN_Web_docs:localStorage}.
\item\texttt{key(n)}, che restituisce il nome della \texttt{n}-esima chiave \cite{MDN_Web_docs:localStorage}. 
\end{itemize}
È definito anche l'evento "\texttt{storage}", che viene lanciato quando il contenuto dello \texttt{Storage} subisce delle modifiche: questo evento può essere catturato da tutte le schede connesse allo \texttt{Storage} diverse dalla pagina che ha modificato il contenuto. Per ascoltare questo evento basta aggiungere un opportuno \texttt{event listener} all'oggetto \texttt{window} \cite{MDN_Web_docs:storage_API}.\\
\texttt{localStorage} e \texttt{sessionStorage} non sono disegnati per contenere una grande quantità di dati: un oggetto di tipo \texttt{Storage}, infatti, ha una capienza massima di 10MB. Se è necessario memorizzare grandi moli di dati bisogna ricorrere a \texttt{IndexedDB} \cite{MDN_Web_docs:IndexedDB_basic}.

\section{IndexedDB} %<---NON È PROGETTATO PER SERVER-SIDE
\textit{\texttt{IndexedDB}} (\textit{Indexed database}) è un'altra strategia con cui le Web App possono immagazzinare dati nel Browser dell'utente. \texttt{IndexedDB} implementa un \textit{Object Oriented Database Management System} (\textit{OODBMS}): in quanto tale, anch'esso memorizza coppie chiave-valore, tuttavia, a differenza di \texttt{localStorage} e \texttt{sessionStorage}, questi due possono essere un qualunque oggetto, non per forza stringhe.\\
Mentre in un \textit{RDBMS} (\textit{Relational Database Management System}) i dati sono memorizzati come righe all'interno di una tabella, in un OODBMS essi sono salvati come oggetti all'interno di \textit{Object Store}: cisacun record all'interno di un Object Store sarà identificato da una \textit{chiave} univoca (come nei RDBMS), che può essere sfruttata per ottenere il riferiìmento all'oggetto stesso.\\
Oltre che con le chiavi, gli Object Store possono essere interrogati anche attraverso indici: un \textit{indice} è un Object Store ausiliario che si riferisce a un altro Object Store, e che serve a effettuare ricerche all'interno di quest'ultimo.
\backmatter

\begingroup %------------------------------ BIBLIOGRAPHY
  \makeatletter
  \let\ps@plain\ps@empty
  \makeatother
  \bibliography{report}
  \addcontentsline{toc}{chapter}{Bibliografia}
  \bibliographystyle{ieeetr} % sort in order of appearance
\endgroup
\end{document} 